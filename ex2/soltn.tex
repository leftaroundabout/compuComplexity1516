
% \newcommand{\vv}{\vec{v}}
% \newcommand{\uv}{\vec{u}}
\newcommand{\unitV}[1]{{\vec{e}_{\mathrm{#1}}}}
\newcommand{\posr}{\vec{r}}
\newcommand{\Force}{\vec{F}}
\newcommand{\AngMom}{\vec{L}}
\newcommand{\mom}{\vec{p}}
% \newcommand{\kwv}{\vec{k}}

\exno{Motion under central force}

Let $\Force : \mathbb{R}^3\to\mathbb{R}^3$ be a central force:
\[
  \Force = F(r)\cdot \unitV{r}
         = \frac{F(r)}{r}\cdot \posr
\]
where $F$ should vanish\footnote{%
Strictly speaking, we do not need this here (the integral definition for $V$ could start from any finite distance as well), but for actual physical forces it is required.
} in the limit $r\to\infty$ fast enough to be integrable.

\subex{}
Let $V : \mathbb{R}^3\to\mathbb{R}$,
\[
  V(r) = \int\limits^r_\infty\ttd r'\: F(r')
\]
then, using spherical coordinates,
\[
  \nabla V
   = \partial_r V(r) \cdot \unitV{r}
    + \frac1r \partial_\vartheta V(r) \cdot \unitV{\vartheta}
    + \frac1{r\sin\theta} \partial_\varphi V(r) \cdot \unitV{\varphi}
   = \unitV{r} \partial_r\int\limits^r_\infty\ttd r'\: F(r')
   = \unitV{r} F(r)
   = \Force
\]
i.e. $\Force$ can be written as gradient of the potential $V$.

\subex{}
The angular momentum of a particle of mass $m$ at position $\posr(t)$ is
\[
  \AngMom(t) = \posr \times m\dot\posr.
\]
It is conserved:
\[
  \partial_t \AngMom
     = (\partial_t\posr) \times m\dot\posr
       + \posr \times \partial_t m\dot\posr
     = \dot\posr \times m\dot\posr
       + \posr \times m\ddot\posr
     = 0 + \posr \times F(r)\unitV{r}
     =  F(r) \posr \times\unitV{r}
     = 0.
\]


\exno{Energy conservation of central force field}

\subex{}
Given any starting configuration of linearly independent%
\footnote{If position and velocity are linearly dependent, then the entire dynamics are \emph{one}-dimensional.}
$(\posr, \dot\posr)$, these vectors span a two-dimensional coordinate frame. Because $\partial_t \dot\posr = \tfrac1m \Force \,\|\, \posr$, the velocity and thus also position will then always stay in this plane, i.e. the remaining space dimension is not actually needed to describe the system dynamics.

Use polar coordinates in the plane:
\[
  \posr = r\cdot (\cos\theta\cdot \unitV{x} + \sin\theta \cdot \unitV{y}).
\]

\subex{}
\[
  \dot\posr(t)
     = \dot r \cdot (\cos\theta\cdot \unitV{x} + \sin\theta \cdot \unitV{y})
      + r \cdot (\partial_t \cos\theta\cdot \unitV{x} + \partial_t\sin\theta \cdot \unitV{y})
     = \frac{\dot r}{r} \cdot \posr
      + r \cdot (-\dot\theta\sin\theta\cdot \unitV{x} + \dot\theta \cos\theta \cdot \unitV{y})
     = \frac{\dot r}{r} \cdot \posr
      + \dot\theta \cdot (\unitV{z}\times\posr)
\]

\subex{}

\[
  \dot\posr = \dot r \unitV{r} + r\dot\theta\unitV{\theta}
\]
\[
  E = \tfrac12 m (\dot\posr)^2 + V
\]
\[
  \frac{2(E - V)}{m} = (\dot\posr)^2 = \dot r^2 + \frac{(r^2\dot\theta)^2}{r^2}
     = \frac{1}{u^4}(\tfrac{\ttd u}{\ttd\theta})^2 \dot\theta^2 + u^2 h^2
\]


\exno{}
The cross-product of linear- and angular momentum varies as follows:
\[
  \partial_t (m\dot\posr \times \AngMom)
    = (\partial_tm\dot\posr) \times \AngMom
     + m\dot\posr \times \underbrace{\partial_t\AngMom}_{\mathclap{\text{conserved}}}
    = m\ddot\posr \times \AngMom
    = \Force \times \AngMom
    = F(r)\unitV{r} \times (\posr \times m\dot\posr)
    = \frac{F(r)}r\cdot m \left(
         \posr (\posr\cdot \dot\posr) - \dot\posr (\posr \cdot \posr)
       \right)
    = \frac{F(r)}r\cdot m \left(
         \posr (r\cdot \dot r) - r^2 \dot\posr
       \right)
    = F(r)\cdot m \left(
          \dot r \posr - r\dot\posr
       \right)
    = F(r)\cdot mr^2 \partial_t\left(
          \tfrac{\posr}{r}
       \right)
\]
where it was used that
\[
  \partial_t\tfrac{\posr}{r}
  = \posr \partial_t\tfrac1r + \tfrac1r \partial_t\posr
  = - \posr\tfrac1{r^2}\dot{r} + \tfrac1r \dot\posr.
\]
If now $F(r) = \frac{\alpha}{r^2}$, then $F(r)\cdot mr^2$ is constant and can thus be
pulled in the derivative:
\[
  \partial_t (m\dot\posr \times \AngMom)
    = \partial_t \left( \alpha\cdot m
          \tfrac{\posr}{r}
       \right)
    = \partial_t ( \alpha\cdot m \unitV{r} )
\]
\[
  \ifnonlif { 0 = \partial_t ( m\dot\posr \times \AngMom - m\alpha \unitV{r} )
       \:{=:} \partial_t \vec{A} }
\]
where $\vec{A}$ is called \emph{Laplace-Runge-Lenz vector}.

Furthermore,
\[
  \vec{A}\cdot\posr + m\alpha r
    = (m\dot\posr\times\AngMom)\cdot\posr - m\alpha\unitV{r}\cdot\posr + m\alpha r
    = \det\begin{pmatrix}m\dot\posr & \AngMom & \posr\end{pmatrix} - m\alpha r + m\alpha r
    = \det\begin{pmatrix}\AngMom & \posr & m\dot\posr \end{pmatrix} - 0
    = \AngMom\cdot(\posr\times m\dot\posr)
    = L^2.
\]
Since $\AngMom$ is conserved, so is $L^2$ and thus also $\vec{A}\cdot\posr + m\alpha r$.



\end{document}

